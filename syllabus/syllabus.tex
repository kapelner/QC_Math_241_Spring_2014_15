\documentclass[12pt]{article}

\usepackage[margin=0.95in]{geometry}
\usepackage{hyperref}
\usepackage{datetime}
\usepackage[auth-sc,affil-sl]{authblk}
\usepackage{color}
\usepackage{placeins}
\usepackage{enumerate}
\definecolor{black}{rgb}{0,0,0}
\definecolor{blue}{rgb}{0,0,0.7}
\newcommand{\inblue}[1]{\color{blue}\textbf{#1} \color{black}}
\definecolor{green}{rgb}{0.133,0.545,0.133}
\newcommand{\ingreen}[1]{\color{green}\textbf{#1} \color{black}}
\definecolor{yellow}{rgb}{1,0.549,0}
\newcommand{\inyellow}[1]{\color{yellow}\textbf{#1} \color{black}}
\definecolor{red}{rgb}{1,0.133,0.133}
\newcommand{\inred}[1]{\color{red}\textbf{#1} \color{black}}
\definecolor{purple}{rgb}{0.58,0,0.827}
\newcommand{\inpurple}[1]{\color{purple}\textbf{#1} \color{black}}
\definecolor{brown}{rgb}{0.55,0.27,0.07}
\newcommand{\inbrown}[1]{\color{brown}\textbf{#1} \color{black}}

\newcommand{\coursewebpage}{\href{https://github.com/kapelner/QC_Math_241_Spring_2015}{course homepage}}

\newcommand{\qu}[1]{``#1''}


\title{MATH 241 Spring 2015 (3 credits) \\ Course Syllabus}

\author[]{Adam Kapelner, Ph.D.}

\affil[]{Queens College, City University of New York}
\settimeformat{ampmtime}
\date{\small document last updated \today ~\currenttime }

\begin{document}
\maketitle

\begin{table}[htp]
\centering
\begin{tabular}{rl}
Instructor & Professor Adam Kapelner \\
Office & 325 Kissena Hall I (64-19 Kissena Blvd between 64 \& 65 Ave) \\
Contact & \url{kapelner@qc.cuny.edu} \\
Section A Time / Loc & Tuesday and Thursday 9:15 - 10:30AM / Kiely 258 \\
Section B Time / Loc & Tuesday and Thursday 12:15 - 1:30PM / Kiely 258 \\
Office Hours / Loc & Tuesday and Thursday 10:30AM - noon / Kissena 325\\
Teaching Assistant & Meylin Cherta-Ceballos \\
TA Tutoring Time / Loc & TBA / Kiely 331 \\
Course Homepage & \href{https://github.com/kapelner/QC_Math_241_Spring_2015}{https://github.com/kapelner/QC\_Math\_241\_Spring\_2015} \\
\end{tabular}
\end{table}

\section*{Course Overview}

MATH 241 is an introduction to the basic concepts and techniques of probability and statistics with an emphasis on applications. Topics to be covered are below (not in order of coverage):

\begin{itemize}
\itemsep -0.0em 
\item Basic Set Theory
\item Counting Methods --- permutations and combinations
\item Basic Probability Theory --- axioms, conditional probability, in/dependence
\item Modeling with Discrete Random Variables: Bernoulli, Hypergeometric, Binomial, Poisson, Geometric, Negative Binomial, Uniform Discrete, Rademacher and others
\item Expectation, Variance, Covariance, Moments
\item Modeling with Continuous Random Variables: Exponential, Uniform and Normal
\item Moment Generating Functions
\item Law of Large Numbers and the Central Limit Theorem
\item Confidence Intervals and Hypothesis Testing for one-sample proportions (possibly $p$-values and statistical significance as well)
\end{itemize}

Students taking this course may not receive credit for MATH 114, except by permission of the chair. Pre/corequisites include MATH 132 or 143 or 152.

This is not your typical mathematics course. This course develops ideas for helping to make decisions based on statistical concepts as well as modeling real-world situations. The course does not dwell on the details of computation but will make use of computation especially using the \texttt{R} statistical language.


\section*{Course Materials}

\paragraph{Textbook:} A First Course in Probability by Sheldon Ross. I prefer the 7th edition which you can buy this used on \href{http://www.amazon.com/First-Course-Probability-7th-Edition/dp/0131856626}{Amazon}. You can buy any edition though if you find it cheaper. There is no excuse not to have this book. It is \textit{required}. However, most of the material in the class comes from the lecture notes. The textbook is a way to get ``another take'' on the material.

\paragraph{Computer Software:} We will also be using \texttt{R} which is a free, open source statistical programming language and console. You can download it from: \url{http://cran.mirrors.hoobly.com/}. I do not expect you to do \textit{any} programming. I will be giving you \texttt{R} code to run and expect you to interpret the results based on concepts explained during the course.

\paragraph{Calculator:} You can use a TI-84, 85, 89 or any calculator which you wish. I strongly suggest you use \href{http://www.wolframalpha.com/}{Wolfram Alpha} and its smartphone app.

\section*{Announcements}

Announcements will be made via email. I am known to send a couple emails per week on important issues. Thus, I will need the email address that you reliably check. The default is whatever is in CUNYfirst which many of you do not check. (See Homework \#0 for more information).

\section*{Lectures}

I have a no computer / tablet / phone policy during lectures. Only pen / pencil and paper. Classes are 75 minutes and run from Thursday, January 29 until Thursday, May 14. There will be 26 lectures and two days for the two midterm exams which are in class. Exam schedule is given on page~\pageref{subsec:exam_schedule}.


\section*{Homework}

There will be 11--13 homework assignments. Homeworks will be assigned and placed on the \coursewebpage~ and will usually be due a week later in class. Homework will be \textbf{graded} out of 100 with extra credit getting scores possibly $> 100$. I (and the course grader) will be doing the grading. Homework must be printed, neat and stapled (\textbf{it cannot be emailed to me}). Homework can be given to me in class or delivered to my office in Kissena Hall. \textit{Homework cannot be handed in to my mail slot in the Kiely mathematics office} (unless you want it to be counted as late).

Graded homework will be returned in class. Regrades are handled during office hours or right after class is over. Scores for homeworks are finalized one week after the graded copies are handed back. Thereafter there will be no changes and no re-grading. Do not delay checking your graded homeworks. I (and the grader) are not perfect and we do make mistakes. It is your obligation to find our mistakes and report them.

You are encouraged to seek help from me if you have questions. After class and office hours are good times. \ingreen{You are highly recommended to work with each other and help each other.} You must, however, submit your own solutions, \textit{with your own write-up} and in \textit{your own words}. There can be no collaboration on the actual \textit{writing}. Failure to comply will result in severe penalties. The university honor code is something I take seriously and I send people to the Dean every semester for violations.

Homework will be similar to previous semesters' homeworks. I will change questions here and there. If you are copying from a previous students' homeworks, we will eventually find you since the criminal mind eventually will slip. Honesty is the best policy. It's not worth me giving you a zero for your entire homework score.

\subsection*{Philosophy of Homework}


Homework is the \textit{most} important part of this course.\footnote{In one student's \href{http://www.ratemyprofessors.com/ShowRatings.jsp?tid=1951051}{observation}, I give a \qu{mind-blowing homework} every week.} Success in Statistics and Mathematics courses comes from experience in working with and thinking about the concepts. It's kind of like weightlifting; you have to lift weights to build muscles. My job as an instructor is to provide assistance through your \href{http://en.wikipedia.org/wiki/Zone_of_proximal_development}{zone of proximal development}. With me, you can grow more than you can alone. To this effect, homework problems are color coded \ingreen{green} for easy, \inyellow{yellow} for harder, \inred{red} for challenging and \inpurple{purple} for extra credit. You need to know how to do all the greens by yourself. If you've been to class and took notes, they are a joke. Yellows and reds: feel free to work with others. Only do extra credits if you have already finished the assignment.

\subsection*{Time Spent on Homework }

This is a three credit course. Thus, the amount of work outside of the 2.5hr in-class time per week is 6-9 hours. I will aim for 6hr of homework per week on average. However, doing the homework well is your sole responsibility since I will make sure that by doing the homework you will study and understand the concepts in the lectures.

\subsection*{Late Homework}

Late homework will be penalized 10 points per day for a maximum of five days. Do not ask for extensions; just hand in the homework late. After five days, \textbf{you can hand it in whenever you want} until three days before the final, Monday May 18. As far as I know, this is one of the most lenient and flexible homework policies in college. I realize things come up. Do not abuse this policy; you will fall behind.

\subsection*{Homework Bonus Points}

Part of good mathematics is its beautiful presentation. Thus, \ingreen{there will be a 15 point bonus} added to your homework grade  for typing up your homework using the \LaTeX ~typesetting system. I recommend using \href{http://overleaf.com}{overleaf} to write up your homeworks. This has the advantage of (a) not having to install anything on your computer and not having to maintain your \LaTeX ~installation (b) allowing easy collaboration with others (c) alway having a backup of your work since it's always on the cloud. If you insist to have \LaTeX ~running on your computer, you can download it for Windows \href{http://www.miktex.org/download}{here} and for MAC \href{http://www.tug.org/mactex/}{here}. For editing and producing PDF's, I recommend \TeX works which can be downloaded \href{http://www.tug.org/texworks/#Getting_TeXworks}{here}. Please use the \LaTeX ~code provided on the \coursewebpage ~for each homework assignment. Since this is extra credit, do not ask me for help in setting up your computer with \LaTeX~ in class or in office hours.

In addition, there will be many extra credit questions sprinkled throughout the homeworks. They will be worth a variable number of points arbitrarily assigned based on my perceived difficulty of the exercise. Homework scores in the 140's are not unheard of. They are a good boost to your grade.

\subsection*{Homework \#0}

For your first homework (due immediately). You must:

\begin{enumerate}[(1)]
\item email me at \href{kapelner@qc.cuny.edu}{kapelner@qc.cuny.edu} from the email address you wish to be contacted at for this course (most commonly this is a gmail address),
\item in the email, you must say \qu{My name is $<$Your Full Name as appears in the registrar$>$ and I have read and understand all the material in the course syllabus} and
\item in the email, you attach a picture of you so I can memorize and know your name.
\end{enumerate}

I will email you back a password you can use to check the \href{http://kapelner.com/grades}{course grading site} once the site is up (which should be a couple weeks into the semester). \\

This assignment is due Tuesday, Feb 3, 2015 5PM and will receive a grade of 0 or 100 with the usual 10 point penalty for lateness.


\section*{Examinations}

Examinations are solely based on homeworks! If you can do all the green and yellow problems on the homeworks, the exams should not present any challenge. I will \textit{never} give you exam problems on concepts which you have not seen at home on one of the weekly homework assignments. There will be three exams and the schedule is below.

\subsection*{Exam Schedule}\label{subsec:exam_schedule}

\begin{itemize}
\itemsep -0.0em 
\item Midterm examination I will be Thursday March 5 in class
\item Midterm examination II will be Thursday April 16 in class
\item The final examination will be Thursday May 21. Section A has the exam from 8:30-10:30AM in class and section B has the exam from 11AM-1PM in class.
\end{itemize}

\subsection*{Exam Materials}

I allow you to bring any calculator you wish but it cannot be your phone. The only other items allowed are pencil and eraser. I do not recommend using pen but it is allowed

I also allow \qu{cheat sheets} on examinations. For both midterms, you are allowed to bring one 8.5'' $\times$ 11'' sheet of paper (front and back). On this paper you can write anything you would like which you believe will help you on the exam. For the final, you are allowed to bring three 8.5'' $\times$ 11'' sheet of paper (front and back). I will be handing back the cheat sheets so you can reuse your midterm cheat sheets for the final if you wish. 


\subsection*{Cheating on Exams}

If I catch you cheating, you can either take a zero on the exam, or you can roll the dice before the University Honor Council who may choose to suspend you.


\subsection*{Missing Exams}

There are no make-up exams. If you miss the exam, you get a zero. If you are sick, I need documentation of your visit to a hospital or doctor. Expect my grader to call the doctor or hospital to verify the legitimacy of your note. If you need to leave the country for an emergency, I will expect proper documentation as well.

\subsection*{Special Services}

If you are a student who takes exams at the special services center, I need to see your blue slip one week before the exam to make proper arrangements with the center.

\section*{Class Participation (and attendance)}

I will be taking attendance during the class. Attendance counts towards the class participation portion of your grade in equal part with how often you ask and answer questions during the lecture.


\section*{Grading and Grading Policy}

Your course grade will be calculated based on the percentages as follows: 

\begin{table}[h]
\centering
\begin{tabular}{l|l}
Homework & 20\% \\
Class participation & 5\% \\
Midterm Examination I & 20\%\\
Midterm Examination II & 20\%\\
Final Examination & 35\%
\end{tabular}
\end{table}
\FloatBarrier

\noindent Below are tables of previous semesters' grade distributions and approximate cutoffs.

\begin{table}[htp]
\centering
\begin{tabular}{l|cccccccccccc}
Grade & F & D & D+ & C- & C & C+ & B- & B & B+ & A- & A & A+ \\ \hline
$n$ & 6 & 5 & 1 & 5 & 4 & 4 & 4 & 6 & 4 & 2 & 6 & 1 \\
\%ile & 12.5 & 22.9 &  25.0 & 31.3 & 37.5 & 52.1 & 60.4 & 72.9 & 81.3 &  85.4 & 97.9 & 100 \\ \hline
\end{tabular}
\caption{Math 241 Fall, 2014. Total enrollment out of two sections save no-shows was $n=48$.}
\end{table}

I am not obligated to mirror these grade distribution and cutoffs this semester. Grade distributions vary from semester-to-semester based on the relative difficulty of exams, student ability and randomness in raw score clustering. The above is meant for informational purposes only. Do not come to me expecting a negotiation of your grade based on a previous semester's cutoff.



\subsection*{The Computer Science C- requirement}

This class is a required course for the Computer Science major and requires a minimum of a C- to earn credit for the major. I cannot assign grades based on student need. What I can do is let you know if you are in danger of getting a grade lower than C- after the first exam and after the second exam. The second exam is scheduled on the day of the Withdraw deadline and I will do my best to have them graded well before the deadline. 

If you are below the 31.3\%ile and you are a CS major, you are on the border of passing your requirement. If you do not plan on studying hard for the final and improving your grade, you may choose to withdraw, thereby allowing you to focus on other classes only to retake Math 241 in the future and this time earn a stellar grade.

\subsection*{Checking your grade and class standing}

You can always check your grades in real-time using the \href{http://kapelner.com/grades}{grading site}. You will enter in your QC ID number and the password I will provide to you after homework 0.



\section*{Auditing}

Auditors are welcome in both sections. They are encouraged to do all homework assignments. I will even grade them. Note that the university does not allow auditors to take examinations.


\end{document}
